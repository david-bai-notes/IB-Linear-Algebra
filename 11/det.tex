\section{Trace and Determinant}
\begin{definition}
    Let $A\in M_n(F)=M_{n,n}(F)$ be a square $n\times n$ matrix.
    The trace of $A$ is defined to be
    $$\operatorname{tr}A=\sum_{i=1}^nA_{ii}$$
\end{definition}
\begin{remark}
    The map sending a matrix to its trace is a linear form.
\end{remark}
\begin{lemma}
    $\operatorname{tr}(AB)=\operatorname{tr}(BA)$.
\end{lemma}
\begin{proof}
    Write stuff out.
\end{proof}
\begin{corollary}
    Similar matrices have the same trace.
\end{corollary}
\begin{proof}
    $\operatorname{tr}(P^{-1}AP)=\operatorname{tr}(APP^{-1})=\operatorname{tr}(A)$.
\end{proof}
\begin{definition}
    If $\alpha:V\to V$ is linear, then $\operatorname{tr}\alpha=\operatorname{tr}[\alpha]_B$ for any choice of basis $B$ of $V$.
    It is well-defined by the preceding corollary.
\end{definition}
\begin{lemma}
    Let $\alpha:V\to V$ be linear and $\alpha^\ast:V^\ast\to V^\ast$ be the dual map, then $\operatorname{tr}\alpha=\operatorname{tr}\alpha^\ast$.
\end{lemma}
\begin{proof}
    Choose any basis $B$ of $V$, then
    $$\operatorname{tr}\alpha=\operatorname{tr}[\alpha]_B=\operatorname{tr}[\alpha]_B^\top=\operatorname{tr}[\alpha^\ast]_{B^\ast}=\operatorname{tr}\alpha^\ast$$
    as desired.
\end{proof}
Recall that we can decompose any permutation $\sigma\in S_n$ into a product of transpositions.
\begin{definition}
    The signature of a permutation is the (necessarily unique) homomorphism $\epsilon:S_n\to\{1,-1\}$ that sends any transposition to $-1$.
\end{definition}
This map $\epsilon$ is well-defined as we know that the parity of the number of transpositions that builds up a permutation is fixed.
\begin{definition}
    Let $A=(a_{ij})\in M_n(F)$.
    We define the determinant of $A$ as
    $$\det A=\sum_{\sigma\in S_n}\epsilon(\sigma)a_{\sigma(1)1}a_{\sigma(2)2}\cdots a_{\sigma(n)n}$$
\end{definition}
\begin{example}
    For $n=2$, we have
    $$\det\begin{pmatrix}
        a_{11}&a_{12}\\
        a_{21}&a_{22}
    \end{pmatrix}=a_{11}a_{22}-a_{12}a_{21}$$
\end{example}
\begin{lemma}
    If $A=(a_{ij})$ is an upper (resp. lower) triangular matrix, i.e. $a_{ij}=0$ for $i>j$ (resp. $i<j$), then $\det A=0$.
\end{lemma}
\begin{proof}
    The only permutation $\sigma$ such that $\sigma(j)\le j$ (resp. $\sigma(j)\ge j$) for all $j$ is the identity.
\end{proof}
\begin{lemma}
    $\det A=\det A^\top$.
\end{lemma}
\begin{proof}
    For any $\sigma\in S_n$ we know that $\epsilon(\sigma)=\epsilon(\sigma^{-1})$.
\end{proof}
\begin{definition}
    A volumn form $d$ in $F^n$ is a function $(F^n)^n\to F$ such that:\\
    1. It is multilinear:
    For any $i\in\{1,\ldots,n\}$ and $v_1,\ldots,v_{i-1},v_{i+1},\ldots,v_n\in F^n$, the map
    $$v\mapsto d(v_1,\ldots,v_{i-1},v,v_{i+1},\ldots,v_n\in F^n)$$
    is linear.\\
    2. It is an alternating form:
    If $v_i=v_j$ for some $i\neq j$, then $d(v_1,\ldots,v_n)=0$.
\end{definition}
What we want to prove that there is only one volumn form (up to multiplicative constant).
If this is true, then it necessarily equals $\det$ in the following way:
\begin{lemma}
    $\det$ is a volumn form via the obvious identification $M_n(F)=(F^n)^n$ by grouping the $n$ column vectors as a tuple.
\end{lemma}
\begin{proof}
    $\det$ is linear as it is linear in any entry.
    It is an alternating form as $\epsilon$ sends any transposition to $-1$.
\end{proof}
\begin{lemma}
    Let $d$ be a volumn form, then swapping two entries changes the sign.
\end{lemma}
\begin{proof}
    For any $i\neq j$, $d(v_1,\ldots,v_i,\ldots,v_j,\ldots,v_n)+ d(v_1,\ldots,v_j,\ldots,v_i,\ldots,v_n)= d(v_1,\ldots,v_i+v_j,\ldots,v_i+v_j,\ldots,v_n)=0$.
\end{proof}
\begin{corollary}
    For any $\sigma\in S_n$ and volume form $d$,
    $$d(v_{\sigma_1},\ldots,v_{\sigma_n})=\epsilon(\sigma)d(v_1,\ldots,v_n)$$
\end{corollary}
\begin{proof}
    Just decompose $\sigma$ into transpositions.
\end{proof}
\begin{theorem}
    Let $A\in M_n(F)$ and let $A^{(i)}$ be the $i^{th}$ column of $A$.
    For any volumn form $d$, we have
    $$d(A^{(1)},\ldots,A^{(n)})=\det(A)d(e_1,\ldots,e_n)$$
    where $(e_i)_j=\delta_{ij}$.
\end{theorem}
This is what we wanted.
\begin{proof}
    Just expand using linearity and the preceding corollary.
\end{proof}
\begin{corollary}
    $\det$ is the unique volumn form that maps $(e_1,\ldots,e_n)$ to $1$.
\end{corollary}
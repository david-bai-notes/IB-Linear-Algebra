\section{Spectral Theory}
Spectral Theory is the study of spectrum (eigen-stuff) of operators, which is very important in boths maths and physics.
Fix an inner product space $V$.
Recall that an opeator $\alpha\in L(V)$ is self-adjoint if $\alpha=\alpha^\ast$.
\begin{lemma}
    A self-adjoint opeator $\alpha\in L(V)$ has real eigenvalues and eigenvectors with different eigenvalues are orthogonal.
\end{lemma}
\begin{proof}
    If $v\neq 0$ and $\alpha(v)=\lambda v$, then $\lambda\|v\|^2=\langle \lambda v,v\rangle=\langle\alpha(v),v\rangle=\langle v,\alpha(v)\rangle=\langle v,\lambda(v)\rangle=\bar\lambda\|v\|^2$, so $\lambda=\bar\lambda\implies\lambda\in\mathbb R$.\\
    Now if $v,w\neq 0,\lambda\neq\mu$ have $\alpha(v)=\lambda v,\alpha(w)=\mu w$, then $\lambda,\mu\in\mathbb R$ and hence $\lambda\langle v,w\rangle=\langle\alpha(v),w\rangle=\langle v,\alpha(w)\rangle=\langle v,\mu w\rangle=\bar\mu\langle v,w\rangle=\mu\langle v,w\rangle\implies\langle v,w\rangle=0$ since $\lambda\neq\mu$.
\end{proof}
\begin{theorem}[Spectral Theorem for Self-Adjoint Operators in Finite Dimensions]
    Let $V$ be a finite dimensional inner product space over $F=\mathbb R$ or $\mathbb C$ and $\alpha\in L(V)$ is self-adjoint.
    Then $V$ has an orthogonal basis of eigenvectors of $\alpha$.
\end{theorem}
Consequently, $\alpha$ can be diagonalised in an orthonormal basis of $V$.
\begin{proof}
    We proceed by induction on $n=\dim V$.
    $n=1$ is trivial.
    Now assume it is true for $n-1$.
    Let $\lambda$ be a root of $\chi_A$ (exists by FTA).
    Now $\lambda\in\mathbb R$ by the preceding lemma.
    Choose $v\in V\setminus\{0\}$ such that $\alpha(v)=\lambda v$.
    By normalising we can assume $v$ is unit.
    Let $U=\langle \{v\}\rangle^\perp\le V$.
    Then obviously $\alpha(U)\le U$ since $\langle\alpha(u),v\rangle=\langle u,\alpha(v)\rangle=\langle u,\lambda v\rangle=\bar\lambda\langle u,v\rangle=0$ for any $u\in U$.
    Also $\dim U=n-1$.
    Adding $v$ to the basis of $U$ in the induction hypothesis completes the induction process.
\end{proof}
\begin{corollary}
    If $V$ is a finite dimensional inner product space and $\alpha\in L(V)$ is self-adjoint, then $V$ is the orthogonal direct sum of all the eigenspaces of $\alpha$.
\end{corollary}
\begin{proof}
    Follows directly.
\end{proof}
Recall that $\alpha\in L(V)$ is an isometry iff $\alpha^\ast\circ\alpha=\operatorname{id}$.
It is called unitary when $V$ is a vector space over $\mathbb C$.
\begin{lemma}
    Let $V$ be a complex inner product space and $\alpha\in L(V)$ be unitary.
    Then:\\
    (i) All eigenvalues of $\alpha$ are in the unit circle $S^1$.\\
    (ii) Eigenvectors with distinct eigenvalues are orthogonal.
\end{lemma}
\begin{proof}
    If $v\neq 0,\alpha(v)=\lambda v$ we have $\lambda\neq 0$ as $\alpha\neq 0$ and that $\lambda\|v\|^2=\langle\alpha(v),v\rangle=\langle v,\alpha^{-1}(v)\rangle=\langle v, \lambda^{-1}(v)\rangle=\bar{\lambda}^{-1}\|v\|^2$, so $\lambda\bar\lambda=1\implies\lambda\in S^1$.\\
    Now if $v,w\neq 0$ and $\alpha(v)=\lambda v,\alpha(w)=\mu v$ for $\lambda\neq\mu$, then $\lambda\langle v,w\rangle=\langle\alpha(v),w\rangle=\langle v,\alpha^{-1}(w)\rangle=\langle v,\mu w\rangle=\bar{\mu}^{-1}\langle v,w\rangle=\mu\langle v,w\rangle\implies\langle v,w\rangle=0$.
\end{proof}
\begin{theorem}[Spectral Theorem for Unitary Operators in Finite Dimensions]
    Let $V$ be a finite dimensional inner product space over $F=\mathbb C$ and $\alpha\in L(V)$ is unitary.
    Then $V$ has an orthogonal basis of eigenvectors of $\alpha$.
\end{theorem}
Equivalently, $\alpha$ can be diagonalised in an orthonormal basis.
\begin{proof}
    Same idea as in the case for self-adjoint operators.
\end{proof}
Sadly, we cannot tell the same tale for real orthogonal matrices since it can have complex eigenvalues.
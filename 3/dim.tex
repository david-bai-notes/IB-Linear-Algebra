\section{Basis, Dimension and Direct Sum}
\begin{definition}
    This cardinality is called the dimenion $\dim V$ of $V$.
\end{definition}
This is well-defined due to Corollary \ref{dim_well_defined}.
\begin{proposition}
    Let $U,W$ be subspaces of $V$.
    If they are finite dimensional, then so is $U+W$ and
    $$\dim(U+W)=\dim U+\dim V-\dim(U\cap W)$$
\end{proposition}
\begin{proof}
    Pick a basis $v_1,\ldots,v_l$ of $U\cap W$.
    Extend it to a basis $v_1,\ldots,v_l,u_1,\ldots,u_m$ of $U$ and a basis $v_1,\ldots,v_l,w_1,\ldots,w_n$ of $W$, then $\{v_i\}\cup\{u_i\}\cup\{w_i\}$ is easily a basis of $U+W$.
    The equality follows.
\end{proof}
\begin{proposition}
    If $V$ is a finite dimensional vectyor space and $U\le V$, then $U,V/U$ are both finite dimensional and $\dim V=\dim U+\dim V/U$.
    Furthermore,
    $$\dim V=\dim U+\dim V/U$$
\end{proposition}
\begin{proof}
    It is obvious that $U$ is finite dimensional.
    Choose a basis $u_1,\ldots,u_l$ and extend it to a basis $u_1,\ldots,u_l,w_{l+1},\ldots,w_n$ of $V$, then $w_{l+1}+U,\ldots,w_n+U$ can be verified to be a basis of $V/U$.
    The statement is immediate.
\end{proof}
\begin{remark}
    If $U$ is a proper subspace of $V$, written $U<V$ (meaning that $U\le V$ and $U\neq V$), then the proposition gives us $\dim V/U\neq 0$, so $\dim U<\dim V$.
\end{remark}
\begin{definition}
    Let $V$ be a vector space and $U,W\le V$.
    We say $V$ is the direct sum of $U,W$, written $V=U\oplus W$, if every element $v\in V$ can be written uniquely as $v=u+w$ for $u\in U,w\in W$.\\
    If this happens, then we say $W$ is a direct complement of $U$ in $V$.
\end{definition}
Note that direct complement is not unique in general.
\begin{example}
    Take $U=\mathbb R\times \{0\}$, then both $W=\{0\}\times \mathbb R$ and $W'=\langle\{(1,1)^\top\}\rangle$ are direct complements of $U$.
\end{example}
\begin{lemma}
    Let $U,W\le V$, then the followings are equivalent:\\
    1. $V=U\oplus W$.\\
    2., $V=U+W$ and $U\cap W=\{0\}$.\\
    3. For any basis $B_1$ of $U$ and $B_2$ of $W$, the union $B=B_1\cup B_2$ is a basis of $V$.
\end{lemma}
\begin{proof}
    Trivial.
\end{proof}
\begin{definition}
    Let $V_1,\ldots,V_l\le V$, then we define
    $$\sum_{i=1}^lV_i=\{v_1+\cdots+v_l:v_j\in V_j,1\le j\le l\}$$
    The sum is direct, i.e.
    $$\sum_{i=1}^lV_i=\bigoplus_{i=1}^lV_i$$
    iff $v_1+\cdots +v_l=v_1'+\cdots +v_l'$ implies $v_j=v_j'$ for any $1\le j\le l$ and $v_j\in V_j$.
    Equivalently,
    $$V=\bigoplus_{i=1}^lV_i\iff \forall v\in V,\exists!(v_1,\ldots,v_l)\in V_1\times\cdots\times V_l,v=\sum_{i=1}^lv_i$$
\end{definition}
\begin{proposition}
    The followings are equivalent:\\
    1.
    $$\sum_{i=1}^lV_i=\bigoplus_{i=1}^lV_i$$
    2. For any $i$,
    $$V_i\cap\left( \sum_{j\neq i}V_j \right)=\{0\}$$
    3. For any bases $B_i$ of $V_i$, the union $\bigcup_iB_i$ is a basis of $\sum_iV_i$.
\end{proposition}
\begin{proof}
    Trivial.
\end{proof}
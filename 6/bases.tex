\section{Change of Basis and Equivalent Matrices}
Consider vector spaces $V,W$ and $B=\{v_1,\ldots,v_n\}, B'=\{v_1',\ldots,v_n'\}$ bases of $V$, $C=\{w_1,\ldots,w_n\},C'=\{w_1',\ldots,w_n'\}$.
Let $\alpha:V\to W$ be a linear map.
We want to study the relationship between $[\alpha]_{B,C}$ and $[\alpha]_{B',C'}$.
\begin{definition}
    For a vector space $V$ with bases $B=\{v_1,\ldots,v_n\}, B'=\{v_1',\ldots,v_n'\}$, the change of basis matrix from $B'$ to $B$ is $P=(p_{ij})_{1\le i,j\le n}$ given by $p_{ij}=([v_j']_B)_i$.
\end{definition}
Indeed $P=[\operatorname{id}_V]_{B',B}$.
\begin{lemma}
    $[v]_B=P[v]_{B'}$.
\end{lemma}
\begin{proof}
    $P[v]_{B'}=[\operatorname{id}_V]_{B',B}[v]_{B'}=[\operatorname{id}_V(v)]_B=[v]_B$.
\end{proof}
\begin{remark}
    Let $P$ be the change of basis matrix from $B'$ to $B$, then $P$ is invertible and $P^{-1}$ is the change of basis matrix from $B'$ to $B$.
\end{remark}
So for the problem we stated at the start of this section, we write $P=[\operatorname{id}_V]_{B',B}$ and $Q=[\operatorname{id}_W]_{C',C}$, then
\begin{proposition}
    Let $A=[\alpha]_{B,C}$, $A'=[\alpha]_{B',C'}$ and $P,Q$ be as above, then $A'=Q^{-1}AP$.
\end{proposition}
\begin{proof}
    For any $v\in V$, we evaluate
    $$AP[v]_{B'}=[\alpha]_{B,C}[v]_B=[\alpha(v)]_C=Q[\alpha(v)]_{C'}=Q[\alpha]_{B',C'}[v]_B=QA'[v]_{B'}$$
    So $AP=QA'$, which means $A'=Q^{-1}AP$.
\end{proof}
\begin{definition}[Equivalent Matrices]
    Two matrices $A,A'\in M_{m,n}(F)$ are equivalent if $A'=Q^{-1}AP$ for some $Q\in M_{m,m},P\in M_{n,n}$ invertible.
\end{definition}
\begin{remark}
    As one can check, this defines an equivalent relation.
\end{remark}
\begin{proposition}\label{eqv_form}
    Let $V,W$ be vector spaces over $F$ and $\dim V=n,\dim W=m$.
    Let $\alpha:V\to W$ be linear.
    Then there exists bases $B$ of $V$ and $C$ of $W$ such that
    $$[\alpha]_{B,C}=\left( \begin{array}{c|c}
        I_r&0\\
        \hline
        0&0
    \end{array} \right)$$
    where $I_r$ is the identity matrix of dimension $r=n-n(\alpha)$.
\end{proposition}
\begin{proof}
    Fix a basis $v_{r+1},\ldots,v_n$ of $\ker\alpha$ and extend it to a basis $B=\{v_1,\ldots,v_r\}$.
    It is easy to see that $\alpha(v_1),\ldots,\alpha(v_r)$ gives a basis of $\operatorname{Im}\alpha$ as it is a spanning set that has the right size and that
    $$\sum_{i=1}^r\lambda_i\alpha(v_i)=0\implies \alpha\left( \sum_{i=1}^r\lambda_iv_i \right)=0\implies\sum_{i=1}^r\lambda_iv_i\in\ker\alpha\implies \forall i,\lambda_i=0$$
    Extend this to a basis $C$ of $W$, then $[\alpha]_{B,C}$ can be easily seen to have the right form.
\end{proof}
\begin{remark}
    This provides another proof of the Rank-Nullity Theorem.
\end{remark}
\begin{corollary}
    Any $m\times n$ matrix is equivalent to a matrix in the form illustrated in the preceding proposition.
\end{corollary}
\begin{proof}
    Immediate.
\end{proof}
\begin{definition}
    Let $A\in M_{m\times n}(F)$.
    The column rank $r(A)$ of $A$ is the dimension of the subspace spanned by the columns of $A$ in $F^n$.
    Similarly, the row rank of $A$ is the column rank of $A^T$.
\end{definition}
\begin{remark}
    If $\alpha$ is a linear map represented by $A$ with respect to some basis, then $r(\alpha)=r(A)$.
\end{remark}
\begin{proposition}
    Two matrices $A,A'$ of the same dimension are equivalent iff $r(A)=r(A')$.
\end{proposition}
\begin{proof}
    Direct consequence of Proposition \ref{eqv_form} and the preceding remark.
\end{proof}
\begin{theorem}
    $r(A)=r(A^T)$.
\end{theorem}
\begin{proof}
    Let $r=r(A)$, so there are some invertible $Q,P$ of the right sizes such that
    $$Q^{-1}AP=\left( \begin{array}{c|c}
        I_r&0\\
        \hline
        0&0
    \end{array} \right)$$
    which is an $m\times n$ matrix.
    But then
    $$P^\top A^\top (Q^{\top})^{-1}=P^\top A^\top(Q^{-1})^\top=(Q^{-1}AP)^\top=\left( \begin{array}{c|c}
        I_r&0\\
        \hline
        0&0
    \end{array} \right)$$
    as an $n\times m$ matrix.
    So $r(A^\top)=r=r(A)$.
\end{proof}
In the case where $\alpha:V\to V$ is an endomomrphism, the change of basis formula becomes $A'=P^{-1}AP$ where $P$ is the change of basis matrix and $A',A$ are the matrices of $\alpha$ in the two different bases.
This induces the following definition.
\begin{definition}
    Let $A,A'$ be square matrices.
    We say $A,A'$ are similar (or conjugate) if there is an $n\times n$ invertible square matrix $P$ such that $A'=P^{-1}AP$
\end{definition}
This notion is central to the study of diagonalisation and spectral theory.
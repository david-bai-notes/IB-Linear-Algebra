\section{Elementary Equations and Elementary Matrices}
\begin{definition}
    An elementary column operation on an $m\times n$ matrix $A$ are one of the followings:\\
    (i) Swap columns $i,j$ with $i\neq j$.\\
    (ii) Multiply the entire column $i$ by $\lambda\in F\setminus\{0\}$.\\
    (iii) Add $\lambda$ times column $i$ to column $j$ where $\lambda\in F$.
\end{definition}
We can do row operations in a analogous (transposed) way.
Something remarkable is that these operations are invertible.
Instead of find the inverses one-by-one, we realise these operations via the action of elementary matrices.
Let $E_{ij}$ be the matrix with $1$ on the $i,j$ entry and zero anywhere else, we have:
\begin{definition}[Elementary Matrices]
    The elementary matrices are $T_{ij}=I-E_{ii}-E_{jj}+E_{ij}+E_{ji}$ for $i\neq j$, $M_{i,\lambda}=I+(\lambda-1)E_{ij}$ for $\lambda\neq 0$ and $C_{i,j,\lambda}=I+\lambda E_{ij}$.
\end{definition}
Then, we easily see that $T_{ij}$ corresponds to column (row) operation (i), $M_{i,\lambda}$ to operation (ii) and $C_{i,j,\lambda}$ to operation (iii) via the operation of multiplying $A$ with the correspondinng matrix from the right (left).
\begin{example}
    $$\begin{pmatrix}
        1&2\\
        3&4
    \end{pmatrix}\begin{pmatrix}
        0&1\\
        1&0
    \end{pmatrix}=\begin{pmatrix}
        2&1\\
        4&3
    \end{pmatrix}$$
\end{example}
\begin{proof}[Constructive Proof of Proposition \ref{eqv_form}]
    It suffices to show that we can get from any matrix to
    $$\left( \begin{array}{c|c}
        I_r&0\\
        \hline
        0&0
    \end{array} \right)$$
    via elementary column and row operations.
    Start with a matrix $A$.
    If $A=0$ then we are done.
    Otherwise pick $a_{ij}=\lambda\neq 0$ and swap rows $i$ and $1$ and then columns $j$ and $1$, after which $\lambda$ is at position $1,1$.
    Then multiply column $1$ by $\lambda^{-1}$.
    So we get $1$ at position $1,1$.
    Now we clean up row $1$ and column $1$ via operation (iii) (both row and column).
    Afterwards we can perform the same procedure on the submatrix by removing the first row and column.
    By induction we can get the desired form at the end.
\end{proof}
There are a few variations on these row and column operations.
The first one is Gauss' pivot algorithm.
If one use only row operations, then one will reach the ``row echelon form'' (which we will define later) in the following way:
Assume $a_{i1}\neq 0$ for some $i$.
Otherwise just move on by deleting the first zero columns.
Then swap rows $i$ and $1$ and divide row $1$ by $\lambda=a_{i1}$ to get $1$ at position $1,1$ and use (iii) to clean up the first column.
And one can move on with the same method to the submatrix removing the first row and first column.
Do this repeatedly and at the end we can get a matrix satisfying:\\
1. For any $i$ here exists $k(i)\ge i$ such that $a_{ij}=0$ for any $j<k(i)$.\\
2. $k(i)$ is increasing in $i$.\\
3. Row $k(i)$ equals $e_{k(i)}$.\\
And matrices satisfying these conditions are called matrices in row echelon form.
Note that the operations above is exactly what we will get when solving a linear system of equations.
So this can be an algorithm of doing that, which is now known as Gauss' pivot algorithm (or Gaussian elimination).\\
Another variation is the following:
\begin{lemma}
    We can obtain the identity matrix from any invertible square matrix via column operations only.
\end{lemma}
By transpose, we can replace column operations by row operations.
\begin{proof}
    We argue by induction on the $k$ where we can guarantee to transform $A$ to the form
    $$\begin{pmatrix}
        I_k&0\\
        \ast&\ast
    \end{pmatrix}$$
    The initial case is obvious.
    Suppose we can do this for some $k$, we shall show that we can do this for $k+1$.
    Now there must be some $j>k$ such that $a_{k+1,j}=\lambda\neq0$.
    Otherwise the vector $e_{k+1}$ is not in the span of the column vectors of $A$, contradiction.
    So we swap columns $k+1$ and $j$ and then divide column $k+1$ by $\lambda$.
    This gets us $1$ at position $k+1,k+1$.
    Then we can clean up the row $k+1$ by this $1$, which completes the induction process.
\end{proof}
This immediately provides an algorithm (a quite cost-effective one) for computing the inverse.
As one can see, this algorithm is analogous to the algorithm of solving a nonsingular linear system.
Also,
\begin{proposition}
    Any invertible matrix is a product of elementary matrices.
\end{proposition}
\begin{proof}
    Writing the operations in the preceding lemma as a product of elementary matrices representing the operations gives the inverse of that matrix.
    But any invertible matrix is the inverse of its own inverse.
\end{proof}
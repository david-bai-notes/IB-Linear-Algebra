\section{Dual Space and Dual Maps}
\begin{definition}
    Let $V$ be a vector space over $F$, we define the dual space $V^\ast$ of $V$ to be the set of linear maps from $V$ to $F$, i.e. $V^\ast=L(V,F)$.\\
    We call linear maps $V\to F$ as linear forms.
\end{definition}
\begin{example}
    1. The map $\operatorname{tr}:M_{n,n}(F)\to F$ via $A=(a_{ij})\mapsto\sum_ia_{ii}$ is a linear form, so $\operatorname{tr}\in (M_{n,n}(F))^\ast$.\\
    2. For a function $f:\in C^{\infty}([0,1],\mathbb R)$, we can define the map $T_f:C^{\infty}([0,1],\mathbb R)$ via
    $$T_f(\phi)=\int_0^1f(x)\phi(x)\,\mathrm dx$$
    An interesting thing is, if we are given all information about $T_f$, can we recover $f$?
    The answer is yes and is left as an exercise.
    This idea comes from quantum mechanics, where you only get the information about $T_f$ by physical measurements but you want information about $f$.
\end{example}
There is a natural way of finding a basis for the dual space.
\begin{lemma}
    Let $V$ be a vector space over $F$ with a finite basis $B=\{e_1,\ldots,e_n\}$, then there exists a basis for $V^\ast$ given by
    $B^\ast=\{\epsilon_1,\ldots,\epsilon_n\}$ where $\epsilon_j(\sum_ia_ie_i)=a_j$ for $1\le j\le n$.
\end{lemma}
\begin{definition}
    The basis $B^\ast$ in the preceding lemma is called the dual basis.
\end{definition}
\begin{remark}
    If we introduce the Kronecker delta
    $$\delta_{ij}=\begin{cases}
        1\text{, if $i=j$}\\
        0\text{, otherwise}
    \end{cases}$$
    then we can define $\epsilon_j$ by extending $\epsilon_j(e_i)=\delta_{ij}$.
\end{remark}
\begin{proof}
    If there is some $\lambda_j\in F$ such that $\sum_j\lambda_j\epsilon_j=0$, then in particular the evaluation of the left hand side at $e_i$ is zero for each $i$.
    But $\epsilon_j(e_i)=0$, so $\lambda_i=0$ for each $i$.\\
    To see this is spanning, one just observe that $\alpha=\sum_j\alpha(e_j)\epsilon_j$ for any linear form $\alpha$.
\end{proof}
\begin{corollary}
    If $V$ is finite dimensional, then $\dim V^\ast=\dim V$.
\end{corollary}
\begin{remark}
    It is convenient to think of $V^\ast$ as the space of row vectors of length $n$ over $F$.
    Indeed, if we have a basis $\{e_i\}$ of $V$ and the corresponding dual basis $\{\epsilon_i\}$ for $V^\ast$, then via calculation we can obtain $\alpha(x)=\sum_i\alpha_ix_i$ where $\alpha=\sum_i\alpha_i\epsilon_i$ and $x=\sum_ix_ie_i$.
\end{remark}
\begin{definition}
    If $U\subset V$ is a subset of the vector space $V$, then the annihilator of $U$ is
    $$U^\circ=\{\alpha\in V^\ast:\forall u\in U,\alpha(u)=0\}$$
\end{definition}
\begin{lemma}
    $U^\circ\le V^\ast$ and if $U\le V$ and $\dim V$ is finite, then $\dim V=\dim U+\dim U^\circ$.
\end{lemma}
\begin{proof}
    Quite obvious that $U^\circ\le V^\ast$.
    To see the identity, write $n=\dim V$ and extend a basis $\{e_1,\ldots,e_k\}$ of $U$ to a basis $\{e_1,\ldots,e_n\}$ of $V$.
    Let $\{\epsilon_1,\ldots,\epsilon_n\}$ be the dual basis.
    Then it becomes obvious that $U^\circ=\langle\{\epsilon_{k+1},\ldots,\epsilon_n\}\rangle$, which gives the identity.
\end{proof}
\begin{lemma}
    Let $V,W$ be vector spaces over $F$ and $\alpha\in L(V,W)$.
    Then the map $\alpha^\ast:W^\ast\to V^\ast$ sending $\epsilon$ to $\epsilon\circ\alpha$ is linear.
\end{lemma}
\begin{proof}
    Obvious.
\end{proof}
\begin{definition}
    This map $\alpha^\ast$ is called the dual map of $\alpha$.
\end{definition}
\begin{proposition}
    Let $V,W$ be finite dimensional vector spaces over $F$ with bases $B,C$.
    Let $B^\ast,C^\ast$ be the dual bases of $V^\ast,W^\ast$, then $[\alpha^\ast]_{C^\ast,B^\ast}=[\alpha]_{B,C}^\top$.
\end{proposition}
\begin{proof}
    Just write it down.
\end{proof}
\begin{lemma}
    Let $E,F$ be bases of $V$ and $P=[\operatorname{id}]_{F,E}$ be the change-of-basis matrix from $F$ to $E$.
    Let $E^\ast,F^\ast$ be the corresponding dual bases, then the change-of-basis matrix from $F^\ast$ to $E^\ast$ is $(P^{-1})^\top$.
\end{lemma}
\begin{proof}
    We have
    $$[\operatorname{id}]_{F^\ast,E^\ast}=[\operatorname{id}]^\top_{E,F}=(P^{-1})^\top$$
    as desired.
\end{proof}
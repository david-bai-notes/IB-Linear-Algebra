\section{Sylvester's Law, Sesquilinear Forms}
We start by looking at some immediate corollaries of \ref{bilinear_diag}.
\begin{corollary}
    For a finite dimensional complex vector space $V$ and a symmetric bilinear form $\phi$ on $V$, there is a basis $B$ of $V$ such that
    $$[\phi]_B=\begin{pmatrix}
        I_r&0\\
        0&0
    \end{pmatrix}$$
    where $r=r(\phi)$.
\end{corollary}
\begin{proof}
    Square roots always exist as $F=\mathbb C$.
\end{proof}
\begin{corollary}
    Every symmetric matrix in $\mathbb C$ is congruent to a unique matrix of the form
    $$\begin{pmatrix}
        I_r&0\\
        0&0
    \end{pmatrix}$$
\end{corollary}
\begin{proof}
    Immediate.
\end{proof}
\begin{corollary}
    If $F=\mathbb R$, $\dim V=n<\infty$ and $\phi$ a symmetric bilinear form of $V$, then there exists a basis $\{v_1,\ldots,v_n\}$ of $V$ such that
    $$\begin{pmatrix}
        I_p&&\\
        &-I_q&\\
        &&0
    \end{pmatrix}$$
\end{corollary}
\begin{proof}
    Every positive number in $\mathbb R$ has a square root.
\end{proof}
\begin{definition}
    $s(\phi)=p-q$ is called the signature of the real symmetric bilinear form $\phi$.
\end{definition}
To see it is well-defined,
\begin{theorem}[Sylvester's Law of Inertia].
    If a real symmetric bilinear form $\phi$ has
    $$[\phi]_B=\begin{pmatrix}
        I_p&&\\
        &-I_q&\\
        &&0
    \end{pmatrix},[\phi]_{B'}=\begin{pmatrix}
        I_{p'}&&\\
        &-I_{q'}&\\
        &&0
    \end{pmatrix}$$
    then $p=p',q=q'$.
\end{theorem}
\begin{definition}
    Let $\phi$ be a real symmetric bilinear form.
    We say that $\phi$is positive semidefinite if $\phi(u,u)\ge 0$ for any $u\in V$, and is positive definite if $\phi(u,u)>0$ for any $u\in V\setminus\{0\}$.
    Similarly, $\phi$ is positive semidefinite if $\forall u\in V,\phi(u,u)\le 0$ for any $u\in V$, and negative definite if $\forall u\in V\setminus\{0\},\phi(u,u)<0$.
\end{definition}
\begin{example}
    The matrix
    $$\begin{pmatrix}
        I_p&0\\
        0&0
    \end{pmatrix}\in M_n(\mathbb R)$$
    is always positive semidefinite, and is positive definite iff $p=n$.
\end{example}
\begin{proof}
    Indeed $p$ is the largest dimension of subspace of $V$ in which $\phi$ is positive definite.
    Similarly $q$ is the largest dimension of a subspace in which $q$ is negative definite.
    These descriptions are independent of the choice of basis, so we are done
\end{proof}
\begin{definition}
    The kernel of the bilinear form $\phi:V\times V\to F$ is the set $K(\phi)=\{v\in V:\forall u\in V,\phi(u,v)=0\}$.
\end{definition}
\begin{remark}
    1. $\dim K+r(\phi)=0$.\\
    2. For $F=\mathbb R$, we now know from the preceding theorem that there is a subspace $T$ of dimension $n-(p+q)+\min{p,q}$ such that $\phi|_T=0$.
    More over, this can easily be shown to be the largest dimension such that such a subspace $T$ exists.
\end{remark}
Recall that the standard inner product on $\mathbb C^n$, that is
$$\langle x,y\rangle=\sum_{i=1}^nx_i\bar{y}_i$$
is not a bilinear form.
\begin{definition}
    Let $V,W$ be vector spaces over $\mathbb C$.
    A map $\phi:V\times W\to\mathbb C$ is a sesquilinear form if for any $w\in W$, $\phi(\cdot,w)$ is linear and for any $v\in V,\lambda_1,\lambda_2\in \mathbb C,w_1,w_2\in W$,
    $$\phi(v,\lambda_1w_1+\lambda_2w_2)=\bar{\lambda}_1\phi(v,w_1)+\bar\lambda_2\phi(v,w_2)$$
\end{definition}
\begin{definition}
    With notation as above, for bases $B=\{v_1,\ldots,v_m\}$ of $V$ and $C=\{w_1,\ldots,w_n\}$ of $W$, the matrix of $\phi$ is $(\phi]_{B,C})_{ij}=(\phi(v_i,w_j))$.
\end{definition}
\begin{lemma}
    $\phi(v,w)=[u]_B^\top [\phi]_{B,C}\overline{[v]}_C$.
\end{lemma}
\begin{proof}
    Expand.
\end{proof}
\begin{lemma}
    If $B,B'$ are bases of $V$ and $C,C'$ of $W$ and $P=[\operatorname{id}_V]_{B',B},Q=[\operatorname{id}_W]_{C',C}$, then $[\phi]_{B',C'}=P^\top[\phi]_{B,C}\bar{Q}$
\end{lemma}
\begin{proof}
    Analogous to the bilinear case.
\end{proof}
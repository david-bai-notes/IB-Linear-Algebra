\section{Some Properties of Determinant}
\begin{lemma}
    Let $A,B$ be sqaure matrices, then $\det(AB)=\det(A)\det(B)$.
\end{lemma}
\begin{proof}
    Direct expansion does the trick, but alternatively we can define $d_A$ via $d_A(B)=\det(AB)$ which is obviously a volume form.
    Therefore $d_A(B)=d_A(I)\det(B)=\det(A)\det(B)$.
\end{proof}
\begin{definition}
    Let $A\in M_n(F)$.
    We say $A$ is singular if $\det A=0$, otherwise we say $A$ is nonsingular.
\end{definition}
\begin{lemma}
    If $A$ is invertible then it is non-singular.
\end{lemma}
\begin{proof}
    Suppose $B$ is an inverse of $A$, then $\det(A)\det(B)=\det(AB)=\det(I)=1\neq 0$ therefore $\det A\neq 0$.
\end{proof}
\begin{remark}
    In particular $\det(A^{-1})=(\det A)^{-1}$ if $A$ is invertible.
\end{remark}
\begin{theorem}
    Let $A\in M_n(F)$, the followings are equivalent:\\
    1. $A$ is invertible.\\
    2. $A$ is non-singular.\\
    3. $r(A)=n$.
\end{theorem}
\begin{proof}
    We just need to show (ii) implies (iii) since we have already done all others.
    Suppose $r(A)<n$, then $\dim\operatorname{span}(A^{(1)},\ldots, A^{(n)})<n$, so there is some $\lambda_1,\ldots,\lambda_n$ not all zero such that $\sum_i\lambda_iA^{(i)}=0$.
    In particular, there is some $j$ such that $\lambda_j\neq 0$ and hence $c_j=-\sum_{i\neq j}(\lambda_i/\lambda_j)A^{(i)}$.
    Expand $\det A$ using multilinearity gives a linear combination of determinants with repeated entries, which is zero as $\det$ is an alternating form.
\end{proof}
\begin{remark}
    By the theorem, it follows easily that the equation $Ax=y$ where $A\in M_n(F),x,y\in F^n$ has a unique solution iff $\det A\neq 0$.
\end{remark}
\begin{lemma}
    Similar matrices have the same determinant.
\end{lemma}
\begin{proof}
    $\det(PAP^{-1})=\det(P)\det(A)\det(P)^{-1}=\det(A)$ for any invertible $P$.
\end{proof}
Therefore the following definition makes sense.
\begin{definition}
    If $\alpha:V\to V$ be an endomorphism, then $\det\alpha=\det[\alpha_{B,B}]$ for any $B$ basis of $V$.
\end{definition}
\begin{theorem}
    $\det:L(V,V)\to F$ satisfies:\\
    1. $\det\operatorname{id}_V=1$.\\
    2. $\det(\alpha\circ\beta)=\det(\beta)\det(\alpha)$.\\
    3. $\det\alpha\neq 0$ iff $\alpha$ is invertible.
    If indeed $\det\alpha\neq 0$, then $\det(\alpha^{-1})=(\det\alpha)^{-1}$.
\end{theorem}
\begin{proof}
    Choose any basis and the rest follows from previous discussions.
\end{proof}
\begin{lemma}
    Let $A\in M_k(F),B\in M_l(F),C\in M_{k,l}(F)$.
    Consider
    $$M_{k+l}(F)\ni M=\left( \begin{array}{c|c}
        A&C\\ \hline
        0&B
    \end{array} \right)$$
    Then $\det M=(\det A)(\det B)$.
\end{lemma}
\begin{proof}
    Write $n=k+l$ and $M=(m_{ij})$.
    Observe that $m_{\sigma(i)i}=0$ if $i\le k$ and $\sigma(i)>k$.
    So for $m_{\sigma(1)1}\cdots m_{\sigma(n)n}\neq 0$, we must have the decomposition $\sigma_1\circ\sigma_2$ such that $\sigma_1$ fixes anything but $1,\ldots, k$ and $\sigma_2$ fixes anything but $k+1,\ldots,n$.
    But then for such $\sigma$, we have $m_{\sigma(j)j}=a_{\sigma_1(j)j}$ for any $j\in\{1,\ldots,k\}$ and $m_{\sigma(j)j}=b_{\sigma_2(s)s}$ where $s=j-k$ for any $j\in\{k+1,\ldots,n\}$.
    Observe also that $\epsilon(\sigma)=\epsilon(\sigma_1)\epsilon(\sigma_2)$, therefore
    \begin{align*}
        \det M&=\sum_{\sigma\in S_n}\epsilon(\sigma)m_{\sigma(1)1}\cdots m_{\sigma(n)n}\\
        &=\sum_{\sigma_1\in S_k,\sigma_2\in S_l}\epsilon(\sigma_1)\epsilon(\sigma_2)a_{\sigma_1(1)1}\cdots a_{\sigma_1(k)k}b_{\sigma_2(1)1}\cdots b_{\sigma_2(l)l}\\
        &=\sum_{\sigma_1\in S_k}\epsilon(\sigma_1)a_{\sigma_1(1)1}\cdots a_{\sigma_1(k)k}\sum_{\sigma_2\in S_l}\epsilon(\sigma_2)b_{\sigma_2(1)1}\cdots b_{\sigma_2(l)l}\\
        &=(\det A)(\det B)
    \end{align*}
    as desired.
\end{proof}
\begin{corollary}
    If $A_1,\ldots,A_k$ are square matrices, then
    $$\det\begin{pmatrix}
        A_1&&&\ast\\
        &A_2&&\\
        &&\ddots&\\
        0&&&A_k
    \end{pmatrix}=(\det A_1)(\det A_2)\cdots (\det A_k)$$
\end{corollary}
\begin{proof}
    Induction on $k$.
\end{proof}
In particular,
$$\det\begin{pmatrix}
    \lambda_1&&&\ast\\
    &\lambda_2&&\\
    &&\ddots&\\
    0&&&\lambda_k
\end{pmatrix}=\lambda_1\cdots\lambda_k$$
\begin{remark}
    Why is it called a volume from?
    Cconsider the volume form $(\mathbb R^3)^3\to F$ via $(a,b,c)\mapsto a\cdot(b\times c)$.
    Then we know that geometrically this gives the (signed) volume of the parallelopiped formed by $a,b,c$.
\end{remark}
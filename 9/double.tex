\section{Properties of the Dual Map and Double Dual}
\begin{lemma}
    Let $V,W$ be vector spaces over $F$ and $\alpha\in L(V,W)$.
    Let $\alpha^\ast\in L(W^\ast,V^\ast)$ be the dual map, then:\\1
    1. $\ker\alpha^\ast=(\operatorname{Im}\alpha)^\circ$, so $\alpha^\ast$ is injective iff $\alpha$ is surjective.\\
    2. $\operatorname{Im}\alpha^\ast\le(\ker\alpha)^\circ$ with equality if $V,W$ are finite dimensional, in which case it implies that $\alpha^\ast$ is surjective iff $\alpha$ is injective.
\end{lemma}
This is very important as it shows how we can understand $\alpha$ from $\alpha^\ast$, which is often simpler.
\begin{proof}
    1. Pick $\epsilon\in W^\ast$, then $\epsilon\in\ker\alpha^\ast$ iff $\alpha^\ast(\epsilon)=0$ iff $\epsilon\circ\alpha=0$ iff $\epsilon\in(\operatorname{Im}\alpha)^\circ$.\\
    2. We first show that $\operatorname{Im}\alpha^\ast\le(\ker\alpha)^\circ$.
    Indeed, for any $\epsilon\in\operatorname{Im}\alpha^\ast$, we have $\epsilon=\alpha^\ast(\phi)=\phi\circ\alpha$ for some $\phi\in W^\ast$.
    But then for any $u\in \ker\alpha$ we have $\epsilon(u)=\phi\circ\alpha(u)=\phi(0)=0$, which means $\epsilon\in(\ker\alpha)^\circ$.
    In finite dimension, pick bases $B,C$ of $V,W$ and we get
    \begin{align*}
        \dim\operatorname{Im}\alpha^\ast&=r(\alpha^\ast)=r([\alpha^\ast]_{C^\ast,B^\ast})=r([\alpha]_{B,C}^\top)\\
        &=r([\alpha]_{B,C})=r(\alpha)\\
        &=\dim V-\dim\ker\alpha\\
        &=\dim (\ker\alpha)^\circ
    \end{align*}
    So they have the same dimension, hence equal.
\end{proof}
We now turn to a very important concept known as double dual.
$V^\ast$ is a vector space too, so we can also construct its dual
$$V^{\ast\ast}=L(V^\ast,F)=(V^\ast)^\ast$$
Why is it important?
Well, not much in finite dimensions, but in infinite dimensonal spaces, it is very hard to find obvious relations between $V$ and $V^\ast$.
However, there is a canonical embedding of $V$ into $V^{\ast\ast}$.
Indeed, pick $v\in V$, consider $\hat{v}:V^\ast\to F$ via $\epsilon\mapsto\epsilon(v)$, which is a well-defined element of $V^{\ast\ast}$.
Quite ironically, our first theorem on this topic is about finite-dimensional spaces.
\begin{theorem}
    If $V$ is finite dimensional, then this operation $\hat{}:V\to V^{\ast\ast}$ we just described is an isomorphism of vector spaces.
\end{theorem}
So we can just identify $V^{\ast\ast}$ with $V$.
\begin{proof}
    Linearity is standard.
    To see it is injective, let $e\in V\setminus\{0\}$ and extend $\{e\}$ to a basis $\{e,e_2,\ldots,e_n\}$ of $V$.
    So the dual basis $(\epsilon,\epsilon_2,\ldots,\epsilon_n)$ would have $\hat{e}(\epsilon)=\epsilon(e)=1$.
    Therefore $\hat{}$ has trivial kernel, hence injective.
    It then follows that it is an isomorphism as $\dim V=\dim V^\ast=\dim V^{\ast\ast}$.
\end{proof}
\begin{remark}
    In further linear analysis and functional analysis, we will see that $\hat{}$ remains injective for a huge class of infinite dimensional vector spaces (those of interests are often space of functions).
    And there are many of them (called reflexive spaces) where $\hat{}$ is actually an isomorphism.
    The theories emerged from here have numerous applications in analysis.
\end{remark}
\begin{lemma}
    Let $V$ be a finite dimensional vector space over $F$ and $U\le V$.
    Define $\hat{U}=\{\hat{u}:u\in U\}\le V^{\ast\ast}$.
    Then $\hat{U}=U^{\circ\circ}=(U^\circ)^\circ$.
\end{lemma}
Thus we can identify $U^{\circ\circ}$ with $U$ too.
\begin{proof}
    Trivial.
\end{proof}
\begin{lemma}
    Let $V$ be finite dimensional vector space over $F$ and $U_1,U_2\le V$, then:\\
    1. $(U_1+U_2)^\circ=U_1^\circ\cap U_2^\circ$.\\
    2. $(U_1\cap U_2)^\circ=U_1^\circ+U_2^\circ$.
\end{lemma}
\begin{proof}
    Just write it out.
\end{proof}